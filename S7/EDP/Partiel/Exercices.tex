\documentclass[12pt,a4paper]{article}
\usepackage[utf8]{inputenc} 
\usepackage[T1]{fontenc}		       
\usepackage{lmodern}			       
\usepackage{babel} 
\usepackage{amsmath}
\usepackage{amsfonts}
\usepackage{amssymb}
\usepackage{graphicx}
\usepackage{xcolor}
\usepackage[text={18cm,27cm},centering]{geometry}

\begin{document}

\begin{center}
    \textbf{Equations aux dérivées partielles}\\
\end{center}

\section*{Problème 1}
On veut résoudre :

\begin{equation}
    \frac{\partial u}{\partial t} - D \frac{\partial^2 u}{\partial x^2} = 0
    \text{ pour } x \in [0, L] \text{ et } t \in [0, T]
\end{equation} 

avec 
$$
\begin{cases}
    u(t, 0) = u(t, L) = 0\\
    u(0, x) = u_0(x)
\end{cases}
$$

\begin{enumerate}
    \item Cherchez toutes les solutions possibles de la forme :
    $u(t, x) = w(t)v(x)$
    
    \item En déduire l'expression de la solution de l'équation (1).
    
    \item Généralisez au cas où les conditions limites ne sont pas homogènes et où on a une solution\dots

\end{enumerate}

\color{blue}
solution

\begin{enumerate}
    \item Soit $u(t, x) = w(t)v(x)$, alors, d'après (1) :
    \begin{align*}
        \frac{\partial u}{\partial t} - D \frac{\partial^2 u}{\partial x^2} = 0
        &\Leftrightarrow 
        w'(t)v(x) - D w(t) v''(x) = 0\\
        &\Leftrightarrow
        \frac{w'(t)}{w(t)} = D \frac{v''(x)}{v(x)}\\
        &\Leftrightarrow
        \exists \lambda \in \mathbb{R},
        \begin{cases}
            \frac{w'(t)}{w(t)} = -\lambda \text{ pour } t \in [0, T]\\
            \frac{-D v''(x)}{v(x)} = \lambda \text{ pour } x \in [0, L]
        \end{cases}\\
        &\Leftrightarrow
        \begin{cases}
            w'(t) = -\lambda w(t), \forall t \in [0, T]\\
            -D v''(x) = \lambda v(x), \forall x \in [0, L]\\
        \end{cases}
    \end{align*}\\


    On s'intéresse d'abord à l'équation en $x$ : $-D v''(x) = \lambda v(x)$.
    \begin{itemize}
        \item \underline{Si $\lambda < 0$ :}
        $$
        v''(x) + \frac{\lambda}{D} v(x) = 0
        \Leftrightarrow
        v(x) = \alpha \mathrm{e}^{\sqrt{\frac{\lambda}{D}} x} + \beta \mathrm{e}^{-\sqrt{\frac{\lambda}{D}} x}\\
        $$
        On utilise les conditions limites :
        \begin{align*}
            \begin{cases}
                v(0) = 0\\
                v(L) = 0
            \end{cases}
            &\Leftrightarrow
            \begin{cases}
                \alpha + \beta = 0\\
                \alpha \mathrm{e}^{\sqrt{\frac{\lambda}{D}} L} + \beta \mathrm{e}^{-\sqrt{\frac{\lambda}{D}} L} = 0
            \end{cases}\\
            &\Leftrightarrow
            \begin{cases}
                \alpha = -\beta\\
                \alpha \mathrm{e}^{2\sqrt{\frac{\lambda}{D}} L} = -\beta
            \end{cases}\\
            &\Leftrightarrow
            \begin{cases}
                \alpha = 0\\
                \beta = 0
            \end{cases}
        \end{align*}
        Donc pas de solutions.

        \item \underline{Si $\lambda = 0$ :}
        $$
        v''(x) = 0
        \Leftrightarrow
        v(x) = \alpha x + \beta
        $$
        On utilise les conditions limites :
        $$
        \begin{cases}
            v(0) = 0\\
            v(L) = 0
        \end{cases}
        \Leftrightarrow
        \begin{cases}
            \beta = 0\\
            \alpha L = 0
        \end{cases}\\
        \Leftrightarrow
        \begin{cases}
            \alpha = 0\\
            \beta = 0
        \end{cases}
        $$
        Donc pas de solutions.

        \item \underline{Si $\lambda > 0$ :}
        $$
        v''(x) + \frac{\lambda}{D} v(x) = 0
        \Leftrightarrow
        v(x) = \alpha \cos\left(\sqrt{\frac{\lambda}{D}} x\right) + \beta \sin\left(\sqrt{\frac{\lambda}{D}} x\right)
        $$
        On utilise les conditions limites :
        \begin{align*}
            \begin{cases}
                v(0) = 0\\
                v(L) = 0
            \end{cases}
            &\Leftrightarrow
            \begin{cases}
                \alpha = 0\\
                \beta \sin\left(\sqrt{\frac{\lambda}{D}} L\right) = 0
            \end{cases}\\
            &\Leftrightarrow
            \begin{cases}
                \alpha = 0\\
                \sin\left(\sqrt{\frac{\lambda}{D}} L\right) = 0
            \end{cases}\\
            &\Leftrightarrow
            \begin{cases}
                \alpha = 0\\
                \sqrt{\frac{\lambda}{D}} L = n\pi, n \in \mathbb{N}^*
            \end{cases}\\
            &\Leftrightarrow
            \begin{cases}
                \alpha = 0\\
                \lambda = \frac{n^2 \pi^2 D}{L^2}, n \in \mathbb{N}^*
            \end{cases}
        \end{align*}\\

        Donc les solutions sont :
        \begin{equation}
            v_n(x) = \alpha \sin\left(\frac{n\pi}{L} x\right), n \in \mathbb{N}^*, \alpha \in \mathbb{R}^*
        \end{equation}


    \end{itemize}
    
\end{enumerate}

\end{document}