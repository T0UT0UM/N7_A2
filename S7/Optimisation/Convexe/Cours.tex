\documentclass[12pt,a4paper]{article}
\usepackage[utf8]{inputenc} 
\usepackage[T1]{fontenc}		       
\usepackage{lmodern}			       
\usepackage{babel} 
\usepackage{amsmath}
\usepackage{amsfonts}
\usepackage{amssymb}
\usepackage{graphicx}
\usepackage{xcolor}
\usepackage{mathtools}
\usepackage{fancyhdr}
\usepackage{enumitem}
\usepackage{tcolorbox}
\usepackage{algorithm2e}
\usepackage[text={15cm,24.5cm},centering]{geometry}

% Définir le texte affiché en fin de page
\pagestyle{fancy}
\fancyhf{}  % Clear the default headers and footers
\rfoot{\hrule
    \vspace{0.3cm}
    \noindent\textsf{Félix de Brandois}
    \hfill \thepage
}
\renewcommand{\headrulewidth}{0pt}

% Style de l'entete
\newcommand{\entete}{
    \noindent\textbf{INSA - ModIA, 4$^e$ année.}
    \hfill \textbf{Années 2023-2024}
    
    \begin{center}
        \textbf{\LARGE Optimisation stochastique}
    \end{center}
}


% Définir la fonction pour créer une boîte de propriété
\newcommand{\propriete}[2]{%
    \begin{tcolorbox}[colback=white,colframe=green!25!white,title=\textbf{Propriété #1}, coltitle=black]
        #2
    \end{tcolorbox}
}

% Définir la fonction pour créer une boîte de définition
\newcommand{\definition}[2]{%
    \begin{tcolorbox}[colback=white,colframe=blue!25!white,title=\textbf{Définition #1}, coltitle=black]
        #2
    \end{tcolorbox}
}

% Définir la fonction pour créer une boîte de théorème
\newcommand{\theoreme}[2]{%
    \begin{tcolorbox}[colback=white,colframe=red!25!white,title=\textbf{Théorème #1}, coltitle=black]
        #2
    \end{tcolorbox}
}




\begin{document}

\entete

\vspace{0.5cm}




\section{Motivation et quelques rappels}

\subsection{Un problème d'optimisation "fréquent"}

\begin{equation}
    \underset{x \in \mathbb{R}^n}{\text{min }} h(x) = \underset{x \in \mathbb{R}^n}{\text{min }} f(x) + g(x)
\end{equation}
avec :
\begin{itemize}
    \item $f : \mathbb{R}^n \rightarrow \mathbb{R}$ lisse, à savoir à gradient lipschitzien :\\
    $\exists L > 0$ tel que $\forall x, y \in \mathbb{R}^n, ||\nabla f(x) - \nabla f(y)|| \leq L ||x - y||$
    \item $g : \mathbb{R}^n \rightarrow \mathbb{R} \cup \{+\infty\}$ convexe, potentiellement non-lisse.\\
\end{itemize}

\textbf{Exemple :}
\begin{enumerate}[label=\roman*)]
    \item $g = 0$ : problème d'optimisation lisse non-convexe.
    \item $g(x) = \lambda ||x||_1$ avec $\lambda > 0$ régularisation parcimonieuse.
    \item Reformulation d'un problème d'optimisation avec contraintes :\\
    $\underset{x \in \mathcal{C}}{\text{min }} f(x)$ avec $\mathcal{C} \subset \mathbb{R}^n$ convexe non-vide.\\
\end{enumerate}


\subsection{Exemples d'applications}

\textbf{Exemple 1 :} Moindres carrés régularisés\\
On dispose d'un modèle linéaire :
$$
\forall x \in \mathbb{R}^n, f(x, \beta) = x^T \beta \quad \quad \beta \in \mathbb{R}^n \text{ paramètre du modèle}
$$

On dispose d'observations $(x_i, y_i) \in (\mathbb{R}^n \times \mathbb{R})^p$ permettant d'estimer $\beta$\\

D'où le problème d'optimisation suivant :\\
\begin{equation}
    \underset{\beta \in \mathbb{R}^n}{\text{min }} \frac{1}{2} \sum_{i=1}^p (f(x_i, \beta) - y_i)^2 \quad \Leftrightarrow \quad \underset{\beta \in \mathbb{R}^n}{\text{min }} \frac{1}{2} ||X\beta - y||_2^2 \quad \text{ avec } X = 
    \begin{pmatrix}
        x_1^T\\
        \vdots\\
        x_p^T
    \end{pmatrix}
    \in \mathbb{M}_{p,n}(\mathbb{R})
\end{equation}


\textit{Texte manquant}\\

\begin{enumerate}[label=\roman*)]
    \item $g(\beta) = \frac{\lambda}{2}||\beta||_2^2$ : régularisation de Tikhonov.
    \item $g(\beta) = \lambda ||\beta||_1$ : régularisation parcimonieuse (LASSO).
    \item $g(\beta) = \frac{\lambda}{2}||\beta||_\beta^2 = $\\
\end{enumerate}

\textit{Texte manquant}\\


\noindent\textbf{Exemple 1 :} SVM (Séparateurs à Vaste Marge)\\
On dispose de données $(x_i)_{i \in \{1, \dots, p\}} \in \mathbb{R}^p$ labelisés $(y_i)_{i \in \{1, \dots, p\}} \in \{-1, 1\}^p$\\
On cherche à construire un hyperplan séparant les données $(x_i)$ selons leurs labels $(y_i)$\\

Dans un premier temps, on suppose qu'il existe un tel hyperplan de vecteurs normal $\beta \in \mathbb{R}^n$, passant par l'origine.\\

\begin{figure}
    \centering
    
    \caption{SVM}
    \label{fig:SVM}
\end{figure}

\noindent\underline{Quel hyperplan choisir ?}\\
$\Rightarrow$ Incertitude : nombre de données, répartition dans $\mathbb{R}^n$, etc...\\
$\Rightarrow$ Maximiser la distance maximale entre les données et l'hyperplan.\\

Condition de séparabilité :
$\quad \forall i \in \{1, \dots, p\}, y_i (x_i^T \beta) \geq 0$\\

D'où le problème d'optimisation suivant :\\
\begin{equation}
    \underset{(\beta, M) \in (\mathbb{R}^n \times \mathbb{R}^+)}{\text{max }} M \quad \text{s.c } \quad \forall i \in \{1, \dots, p\}, \quad y_i \frac{(x_i^T \beta)}{||\beta||_2} \geq M
\end{equation}\\


\noindent\underline{\textbf{Remarque :}} $\qquad d(z, \{\beta^Tx=0\}) = \frac{|\beta^Tz|}{||\beta||_2}$\\
En pratique, la condition de séparabilité n'est pas vérifiée pour tout $i \in \{1, \dots, p\}$\\
$\Rightarrow$ Pénalisation des contraintes non satisfaites.\\
On pose $\forall t \in \mathbb{R}, t^+ = \max(t, 0)$\\

Ceci conduit à formuler un autre problème d'optimisation :\\
\begin{equation}
    \underset{(\beta, M) \in (\mathbb{R}^n \times \mathbb{R}^+)}{\text{max }} M - \lambda \sum_{i=1}^p (1 - y_i \frac{(x_i^T \beta)}{||\beta||_2 M})^+
\end{equation}
avec : $||\beta||_2 = \frac{1}{M}$ et en reformulant pour obtenir un problème de minimisation :\\

\begin{equation}
    \underset{\beta \in \mathbb{R}^n}{\text{min }} ||\beta||_2^2 + \lambda  \sum_{i=1}^p (1 - y_i (x_i^T \beta))^+
\end{equation}

avec : $\sum_{i=1}^p (1 - y_i (x_i^T \beta))^+$ non-lisse (non différentiable)\\



\subsection{Rappels de convexité}

\definition{}{Soit $\mathcal{C} \subset \mathbb{R}^n$. On dit que $\mathcal{C}$ est convexe si :
\begin{center}
    $\forall x, y \in \mathcal{C}$, $\forall \lambda \in [0, 1]$, $\lambda x + (1 - \lambda) y \in \mathcal{C}$
\end{center}
}


\noindent\textbf{Exemples :}
\begin{enumerate}[label=\roman*.]
    \item partie affine : $\{x_0 + s \text{ avec } s \in S\}$ avec $S$ un sous-espace vectoriel de $\mathbb{R}^n$ et $x_0 \in \mathbb{R}^n$.
    \item hyperplan : $\{x \in \mathbb{R}^n \text{ tel que } \alpha^T x = \beta\}$
    \item demi-espace : $\{x \in \mathbb{R}^n \text{ tel que } \alpha^T x \leq \beta\}$
    \item polyèdre : $\{x \in \mathbb{R}^n \text{ tel que } Ax \leq b\}$ avec $A \in \mathbb{M}_{m,n}(\mathbb{R})$ et $b \in \mathbb{R}^m$.
    \item ellipoïde : $\{x \in \mathbb{R}^n \text{ tel que } x^TCx \leq 1\}$ avec $C \in \mathbb{S}_n(\mathbb{R})$ (matrice symétrique semi-définie positive).\\
\end{enumerate}


\propriete{}{Opérations préservant la convexité :
    \begin{itemize}
        \item intersection
        \item somme
        \item multiplication par un scalaire
        \item produit cartésien
        \item image réciproque par une application linéaire
        \item image directe par une application linéaire
        \item projection : $\{x_1 \text{ tel que } (x_1, x_2) \in \mathcal{C}\}$ avec $\mathcal{C}$ convexe.
    \end{itemize}
}

\definition{- Fonctions convexes}{
    Soit $f : \mathcal{C} \rightarrow \mathbb{R}$ avec $\mathcal{C} \subset \mathbb{R}^n$ convexe.
    \begin{itemize}
        \item $f$ est convexe sur $\mathcal{C}$ si : 
        \begin{center}
            $\forall x, y \in \mathcal{C}$, $\forall \lambda \in [0, 1]$, $f(\lambda x + (1 - \lambda) y) \leq \lambda f(x) + (1 - \lambda) f(y)$
        \end{center}
        \item $f$ est strictement convexe sur $\mathcal{C}$ si :
        \begin{center}
            $\forall x, y \in \mathcal{C}$, $\forall \lambda \in ]0, 1[$, $f(\lambda x + (1 - \lambda) y) < \lambda f(x) + (1 - \lambda) f(y)$
        \end{center}
    \end{itemize}
}


\propriete{- CNS}{Condition Nécessaire de convexité dans le cas dérivable.\\
Soit $f : \Omega \rightarrow \mathbb{R}$ avec $\Omega$ ouvert de $\mathbb{R}^n$ et $\mathcal{C} \subset \Omega$ convexe.
\begin{itemize}
    \item Si $f$ est dérivable sur $\Omega$, alors $f$ est convexe sur $\mathcal{C}$ convexe si et seulement si
    \begin{align*}
        &\forall x, y \in \mathcal{C}, f(y) \geq f(x) + f'(x)(y - x)\\
        \Leftrightarrow &\forall x, y \in \mathcal{C}, f(y) \geq f(x) + \nabla f(x)^T(y - x)
    \end{align*}
    \item Si $f$ est deux fois dérivable sur $\Omega$, alors $f$ est convexe sur $\mathcal{C}$ convexe si et seulement si
    \begin{align*}
        &\forall x \in \mathcal{C}, f''(x)(y - x, y - x) \geq 0\\
        \Leftrightarrow &\forall x \in \mathcal{C}, (y - x)^T \nabla^2 f(x) (y - x) \geq 0
    \end{align*}
\end{itemize}
}

\propriete{}{
    \begin{itemize}
        \item $f$ convexe sur $\mathcal{C}$ convexe $\Rightarrow$ $\alpha f$ convexe sur $\mathcal{C}$ convexe pour $\alpha > 0$
        \item Combinaisons linéaires à coefficients positifs de fonctions convexes sont convexes
        \item Soit $f$ convexe sur $\mathcal{C}$ convexe. Soit $A \in \mathbb{M}_{m,n}(\mathbb{R})$ et $b \in \mathbb{R}^m$.\\
        Alors $\mathcal{C}' = \{x \in \mathbb{R}^n \text{ tel que } Ax + b \in \mathcal{C}\}$ est convexe et $x \mapsto f(Ax + b)$ est convexe sur $\mathcal{C}'$.
        \item Soit $(f_i)_{i \in \{1, \dots, m\}}$ convexes sur $(\mathcal{C}_i)_{\{i \in \{1, \dots, m\}\}}$\\
        Alors $\underset{i \in \{1, \dots, m\}}{\text{max }} f_i$ convexe sur $\bigcap_{i=1}^{m} \mathcal{C}_i$.
        \item Soit $g : \mathbb{R}^n \rightarrow \mathbb{R}$ convexe sur $\mathcal{C} \subset \mathbb{R}^n$\\
        Soit $h : \mathbb{R} \rightarrow \mathbb{R}$ croissante et convexe sur $\mathcal{C}'$ tel que $g(\mathcal{C}) \subset \mathcal{C}'$\\
        Alors $h \circ g$ est convexe sur $\mathcal{C}$.
        \item Soit $g : \mathbb{R}^n \rightarrow \mathbb{R}^p$ avec $\forall i \in \{1, \dots, p\}$, $g_i$ convexe sur $\mathbb{R}^n$\\
        Soit $h : \mathbb{R}^p \rightarrow \mathbb{R}$ croissante et convexe vis-à-vis de chacun de ses arguments.
        \vspace{-\baselineskip}\begin{align*}
            \text{Alors } \qquad f : \mathbb{R}^n &\rightarrow \mathbb{R} \qquad \qquad \qquad  \text{ est convexe sur } \mathbb{R}^n \hspace{14.5cm}\\
            x &\mapsto h \circ g(x) = h(g_1(x), \dots, g_p(x))
        \end{align*}
    \end{itemize}
}


\subsection{Régularité des fonctions convexes}

\definition{- Epigraphe}{
    Soit $f : \mathcal{C} \rightarrow \mathbb{R}$\\
    On appelle \textit{épigraphe} de $f$, noté $\varepsilon(f)$, l'ensemble suivant :\\
    $$
    \varepsilon(f) = \{(x, w) \in \mathcal{C} \times \mathbb{R} \text{ tel que } f(x) \leq w\}
    $$
}

\propriete{}{
    Soit $f : \mathcal{C} \rightarrow \mathbb{R}$ avec $\mathcal{C} \subset \mathbb{R}^n$ convexe.\\
    $f$ est convexe sur $\mathcal{C}$ si et seulement si $\varepsilon(f)$ est convexe.
}

\noindent\underline{Preuve :}
\begin{enumerate}[label=\roman*)]
    \item Supposons $f$ convexe sur $\mathcal{C}$ convexe\\
    $\forall x, y \in \mathcal{C}$, $\forall \lambda \in [0, 1]$, $f(\lambda x + (1 - \lambda) y) \leq \lambda f(x) + (1 - \lambda) f(y)$\\
    Soient $(x_1, w_1) \in \varepsilon(f)$ et $(x_2, w_2) \in \varepsilon(f)$ et $\lambda \in [0, 1]$\\
    $f(\lambda x_1 + (1 - \lambda) x_2) \leq \lambda f(x_1) + (1 - \lambda) f(x_2) \leq \lambda w_1 + (1 - \lambda) w_2$\\
    $\Rightarrow \lambda (x_1, w_1) + (1 - \lambda) (x_2, w_2) \in \varepsilon(f)$
    
    \item Supposons $\varepsilon(f)$ convexe\\
    Soit $x, y \in \mathcal{C}$ et $\lambda \in [0, 1]$\\
    $(x, f(x)), (y, f(y)) \in \varepsilon(f)$\\
    $\Rightarrow \lambda (x, f(x)) + (1 - \lambda) (y, f(y)) \in \varepsilon(f)$\\
    $\Rightarrow (\lambda x + (1 - \lambda) y, \lambda f(x) + (1 - \lambda) f(y)) \in \varepsilon(f)$\\
    $\Rightarrow f(\lambda x + (1 - \lambda) y) \leq \lambda f(x) + (1 - \lambda) f(y)$  
\end{enumerate}


\propriete{- Inégalité de Jensen}{
    Soit $f : \mathcal{C} \rightarrow \mathbb{R}$ avec $\mathcal{C} \subset \mathbb{R}^n$ convexe.\\
    Soit $x_1, \dots, x_p \in \mathcal{C}$ et $\lambda_1, \dots, \lambda_p \in \mathbb{R}^+$ tel que $\sum_{i=1}^p \lambda_i = 1$.\\

    Alors $\quad f(\sum_{i=1}^p \lambda_i x_i) \leq \sum_{i=1}^p \lambda_i f(x_i)$
}

\propriete{}{
    Soit $f : \mathcal{C} \rightarrow \mathbb{R}$ avec $\mathcal{C} \subset \mathbb{R}^n$ convexe.
    Soit $x_0 \in \overset{\circ}{\mathcal{C}}$\\
    $f$ est continue en $x_0$.
}


\noindent\underline{Preuve :}\\
Soit $x_0 \in \overset{\circ}{\mathcal{C}}$.
Soit $\Delta$ un simplexe inclus dans $\overset{\circ}{\mathcal{C}}$ et contenant $x_0$.\\
Notons $(s_i)_{i \in \{1, \dots, n + 1\}}$ les sommets de $\Delta$.\\
$\forall x \in \Delta$, $\exists! (\lambda_i)_{i \in \{1, \dots, n + 1\}} \in \mathbb{R}^{n + 1}$ tel que $\lambda_i \geq 0$ et $\sum_{i=1}^{n + 1} \lambda_i = 1$ et $x = \sum_{i=1}^{n + 1} \lambda_i s_i$\\
(coordonnées barycentriques de $x$ vis-à-vis de $\Delta$).\\

\noindent d'où
\begin{align*}
    f(x) &\leq \sum_{i=1}^{n + 1} \lambda_i f(s_i) \text{ par convexité de } f \text{ et inégalité de Jensen.}\hspace{14.5cm}\\
    & \leq \underset{i \in \{1, \dots, n + 1\}}{\text{max }} f(s_i) \text{ car } \sum_{i=1}^{n + 1} \lambda_i = 1
\end{align*}

Donc $f$ est majorée sur $\Delta$.\\
En particulier, $\forall \delta > 0$ tel que $B(x_0, \delta) \subset \Delta$, $f$ est majorée sur $B(x_0, \delta)$.\\
Fixons un tel $\delta$ et notons $M$ un majorant de $f$ sur $B(x_0, \delta)$.\\

\textit{Texte manquant}\\


$\Rightarrow f(\delta) \leq 2f(x_0) - M$\\

Bilan : $2f(x_0) - M \leq f(z) \leq M$, $\forall z \in B(x_0, \delta)$\\
$\Rightarrow$ $f$ est bornée sur $B(x_0, \delta)$\\

Soit $K > 0$ tel que $\forall z \in B(x_0, \delta)$, $|f(z)| \leq K$\\

On montre que $f$ est lipschitzienne sur $B(x_0, \frac{\delta}{2})$\\
Soit $x, y \in B(x_0, \frac{\delta}{2})$, $x \neq y$\\

On pose :
$\begin{cases}
    x' = x - \frac{\delta}{2} \frac{y - x}{||y - x||}\\
    y' = y + \frac{\delta}{2} \frac{y - x}{||y - x||}\\
\end{cases}$\\

Alors $x' \in B(x_0, \delta)$ : $||x' - x_0|| = ||x - \frac{\delta}{2} \frac{y - x}{||y - x||} - x_0|| \leq ||x - x_0|| + \frac{\delta}{2} < \delta$\\
De même, $y' \in B(x_0, \delta)$\\

D'où $|f(x')| \leq K$ et $|f(y')| \leq K$\\

De plus, $x = \frac{2 ||y - x||}{2 ||y - x|| + \delta} x' + \frac{\delta}{2 ||y - x|| + \delta} y = \lambda x' + (1 - \lambda) y'$ avec $\lambda = \frac{2 ||y - x||}{2 ||y - x|| + \delta} \in ]0, 1[$\\

Par convexité de $f$ sur $\mathcal{C}$ : $f(x) \leq \lambda f(x') + (1 - \lambda) f(y)$\\
$\Rightarrow f(x) - f(y) \leq \lambda (f(x') - f(y)) \leq 2K\lambda \leq \frac{4K}{2 ||y - x|| + \delta} ||y - x||$\\
$\Rightarrow |f(x) - f(y)| \leq \frac{4K}{\delta} ||y - x||$\\

De même, $y = \lambda y' + (1 - \lambda) x$ et de la même manière, on montre que $|f(y) - f(x)| \leq \frac{4K}{\delta} ||y - x||$\\
$\Rightarrow |f(x) - f(y)| \leq \frac{4K}{\delta} ||y - x||$\\

Bilan : $f$ est lipschitzienne sur $B(x_0, \frac{\delta}{2})$\\
En particulier, $f$ est continue en $x_0$\\



\section{Sous-différentiell d'une fonction}

\subsection{Sous-gradient et sous différentiel}

\definition{- Sous-differentiel et sous-gradient}{
    Soit $f : \mathcal{C} \rightarrow \mathbb{R}$ avec $\mathcal{C} \subset \mathbb{R}^n$ convexe.
    Soit $x \in \mathcal{C}$ et $g \in \mathbb{R}^n$\\
    $g$ est appelé \textit{sous-gradient} de $f$ en $x$ si :
    \begin{center}
        $\forall y \in \mathcal{C}, f(y) \geq g^T (y - x) + f(x)$
    \end{center}
    On appelle \textit{sous-différentiel} de $f$ en $x$, noté $\partial f(x)$, l'ensemble des sous-gradients de $f$ en $x$ :
    $$
    \partial f(x) = \{g \in \mathbb{R}^n \text{ tel que } \forall y \in \mathcal{C}, f(y) \geq g^T (y - x) + f(x)\}
    $$
}

\noindent\textbf{Exemple :}
$f : \mathbb{R} \rightarrow \mathbb{R}$, $f(x) = |x|$\\
Soit $x \in \mathbb{R}$
\begin{itemize}
    \item si $x < 0$, alors $f(x) = -x$\\
    Soit $g \in \mathbb{R}$ tel que $\forall y \in \mathbb{R}$, $f(y) \geq g(y - x) + f(x)$
    \begin{itemize}
        \item Soit $y \leq 0$, $f(y) = -y = -y + x - x = -(y-x) + f(x) \geq g(y - x) + f(x)$ avec $g = -1$
        \item Soit $y > 0$, $f(y) = y \geq -y + x - x \geq -(y - x) + f(x) \geq g(y - x) + f(x)$ avec $g = -1$
    \end{itemize}
    Donc $\partial f(x) = \{-1\}$
    \item si $x > 0$, $\partial f(x) = \{1\}$ (même raisonnement)
    \item si $x = 0$, $\partial f(x) = [-1, 1]$
\end{itemize}

\propriete{}{
    Soit $f : \mathcal{C} \rightarrow \mathbb{R}$ avec $\mathcal{C} \subset \mathbb{R}^n$ convexe.
    Soit $x \in \mathcal{C}$\\
    Alors $\partial f(x)$ est un convexe non-vide.
}


\noindent\underline{Preuve :}\\
Supposons $\partial f(x) = \emptyset$\\
alors $\partial f(x) = \bigcap_{y \in \mathcal{C}} \{g \in \mathbb{R}^n \text{ tel que } f(y) \geq g^T(y - x) + f(x)\}$\\

or, $\forall y \in \mathcal{C}$, $\{g \in \mathbb{R}^n \text{ tel que } f(y) \geq g^T(y - x) + f(x)\}$ est convexe (demi-espace) et fermé (comme image réciproque d'un fermé par une application continue $\psi$)\\
\begin{flalign*}
    \psi : \mathbb{R}^n &\rightarrow \mathbb{R}\\
    g &\mapsto f(y) - f(x) - g^T(y - x)\\
\end{flalign*}

$\Rightarrow \partial f(x)$ est convexe et fermé comme intersections de parties convexes et fermées.\\


\propriete{}{
    Soit $f : \mathcal{C} \rightarrow \mathbb{R}$ avec $\mathcal{C} \subset \mathbb{R}^n$ convexe.
    Soit $x \in \overset{\circ}{\mathcal{C}}$ tel que $f$ est continue en $x$.\\
    Alors $\partial f(x)$ est borné.
}

\noindent\underline{Preuve :}\\
Soit $x \in \overset{\circ}{\mathcal{C}}$ tel que $f$ est continue en $x$.\\
$x \in \overset{\circ}{\mathcal{C}}$ : $\exists y_1 > 0$ tel que $B(x, y_1) \subset \mathcal{C}$\\
$f$ continue en $x$ : $\forall \varepsilon > 0$, $\exists y_2 > 0$ tel que $\forall y \in B(x, y_2)$, $|f(y) - f(x)| < \varepsilon$\\

Posons $\eta = min(y_1, y_2)$\\

Montrons que $\partial f(x)$ est borné.\\
Supposons le contraire.\\
$\forall M > 0$, $\exists g \in \partial f(x)$ tel que $||g||_2 > M$\\
Soit $M > 0$. Fixons un tel $g$ tel que $||g||_2 > M \Rightarrow g \neq 0$\\

Soit $y = x + \frac{\eta}{2} \frac{g}{||g||_2}$\\
d'où $||y - x||_2 = \frac{\eta}{2} < \eta \Rightarrow y \in B(x, \eta) \subset \mathcal{C}$\\

Par définition de $g$ : $f(y) \geq g^T(y - x) + f(x)$\\
$\Rightarrow f(y) - f(x) \geq \frac{\eta}{2} ||g||_2 > \frac{\eta}{2} M$ avec $M = \frac{2}{\eta}$\\
$\Rightarrow |f(y) - f(x)| > \varepsilon$\\

Or, $y \in B(x, \eta) \Rightarrow y \in B(x, y_2) \Rightarrow |f(y) - f(x)| < \varepsilon$\\

D'où $\varepsilon < |f(y) - f(x)| < \varepsilon$ : contradiction.\\

Donc $\partial f(x)$ est borné.\\


\propriete{}{
    Soit $f : \mathcal{C} \rightarrow \mathbb{R}$ avec $\mathcal{C} \subset \mathbb{R}^n$ convexe.
    \begin{itemize}
        \item $\forall x \in \overset{\circ}{\mathcal{C}}$, $\partial f(x) \neq \emptyset \quad$ (Et $\partial f(x)$ compact convexe non-vide)
        \item Si $f$ dérivable en $x \in \overset{\circ}{\mathcal{C}}$, alors $\partial f(x) = \{\nabla f(x)\}$
    \end{itemize}
}

\noindent\underline{Preuve :}
\begin{enumerate}[label=\roman*)]
    \item Soit $\mathcal{C} \subset \mathbb{R}^n$ convexe non-vide.\\
    Soit $x_0 \in \mathcal{C}^c \cup (\overset{\_}{\mathcal{C}} \text{\textbackslash } \overset{\circ}{\mathcal{C}})$\\
    alors $\exists \alpha \in \mathbb{R}^n \text{\textbackslash} \{0\}$ tel que $\underset{z \in \mathcal{C}}{\text{sup }} \alpha^T z \leq \alpha^T x_0$\\

    \textit{Texte manquant}\\


    Soit $x \in \overset{\circ}{\mathcal{C}}$\\

    \textit{Texte manquant}\\


    \item On suppose de plus $f$ dérivable en $x \in \overset{\circ}{\mathcal{C}}$
    \begin{itemize}
        \item $f$ convexe sur $\mathcal{C}$ convexe et dérivable en $x \in \overset{\circ}{\mathcal{C}}$\\
        $\Rightarrow \forall y \in \mathcal{C}$, $f(y) \geq \nabla f(x)^T(y - x) + f(x)$\\
        $\Rightarrow \nabla f(x) \in \partial f(x)$\\
        $\Rightarrow \{\nabla f(x) \} \subset \partial f(x)$\\

        \item Soit $g \in \partial f(x)$\\
        $\forall y \in \mathcal{C}$, $f(y) \geq g^T(y - x) + f(x)$\\
        Or, $x \in \overset{\circ}{\mathcal{C}}$ : $\exists N \in \mathbb{N} \text{ tel que } y_N = x + \frac{u}{N} \in \mathcal{C}$ avec $u \in \mathbb{R}^n$\\

        Fixons un tel $N$.\\
        $\forall n \geq N$, $f(y_n) \geq \frac{1}{n} g^T u + f(x)$\\

        Or, $f$ dérivable en $x$ :\\
        $f(y_n) = f(x) + \frac{1}{n} \nabla f(x)^T u + \frac{1}{n} ||u||_2 \varepsilon(\frac{1}{n}u)$ avec $\varepsilon(h) \xrightarrow[h \rightarrow 0]{} 0$\\
        $\Rightarrow f(x) + \frac{1}{n} \nabla f(x)^T u + \frac{1}{n} ||u||_2 \varepsilon(\frac{1}{n}u) \geq \frac{1}{n} g^T u + f(x)$\\
        $\Rightarrow (\nabla f(x) - g)^T u + ||u||_2 \varepsilon(\frac{1}{n}u) \geq 0$\\
        
        A la limite : $(\nabla f(x) - g)^T u \geq 0, \forall u \in \mathbb{R}^n$\\
        $\Rightarrow g = \nabla f(x)$\\

        Donc $\partial f(x) \subset \{\nabla f(x)\}$\\

        Bilan : $\partial f(x) = \{\nabla f(x)\}$\\
    \end{itemize}
\end{enumerate}


\propriete{}{
    Soit $f : \mathcal{C} \rightarrow \mathbb{R}$ avec $\mathcal{C} \subset \mathbb{R}^n$ convexe.
    Soit $x^* \in \mathcal{C}$\\
    Alors $x^*$ est un minimum global de $f$ sur $\mathcal{C}$ si et seulement si $0 \in \partial f(x^*)$
}

\noindent\underline{Preuve :}\\
$0 \in \partial f(x^*) \Leftrightarrow \forall y \in \mathcal{C}$, $f(y) \geq 0^T(y - x^*) + f(x^*) = f(x^*) \Leftrightarrow x^*$ est un minimum global de $f$ sur $\mathcal{C}$\\

\color{black}

\subsection{Calculs de sous-gradients}

Pour simplifier, on suppose avoir $(f_i)$ convexes sur $\mathbb{R}^n$\\

\propriete{}{
    \begin{itemize}
        \item Soit $(\alpha_1, \alpha_2) \in (\mathbb{R}_+^*)^2$.
        On pose $f = \alpha_1 f_1 + \alpha_2 f_2$.\\
        Alors $\partial f(x) = \alpha_1 \partial f_1(x) + \alpha_2 \partial f_2(x)$
        \item Soit $h : x \mapsto f(Ax + b)$ avec $A \in \mathbb{M}_{m,n}(\mathbb{R})$ et $b \in \mathbb{R}^m$\\
        Alors $\partial h(x) = A^T \partial f(Ax + b)$
        \item Soit $f : x \mapsto \underset{i \in \{1, \dots, m\}}{\text{max }} f_i(x)$\\
        Soit $I_0 = \{i \in \{1, \dots, m\} \text{ tel que } f_i(x) = f(x)\}$\\
        Alors $\forall g \in \partial f_{I_0}(x)$, $g \in \partial f(x)$
        \item Soit $f : x \mapsto \underset{a \in A}{\text{sup }} f_a(x)$\\
        Soit $I_0 = \{a \in A \text{ tel que } f_a(x) = f(x)\}$\\
        Alors $\forall g \in \partial f_{I_0}(x)$, $g \in \partial f(x)$

        \textit{Texte manquant}\\
    \end{itemize}
}



\section{Algorithmes de sous-gradient}

\subsection{Algorithme du sous-gradient}

On considèle le problème suivant :\\
\begin{equation}
    \underset{x \in \mathbb{R}^n}{\text{min }} f(x) \quad \text{avec } f \text{ convexe sur } \mathbb{R}^n.
\end{equation}

On suppose que $f$ admet au moins un minimum $x^* \in \mathbb{R}^n$\\

On a l'Algorithme suivant :\\
\begin{algorithm}
    \SetAlgoLined
    \textbf{Entrées :} $x_0 \in \mathbb{R}^n$\\
    \textbf{Tant que}\\
    \textit{Texte manquant}\\
\end{algorithm}


\noindent\underline{\textbf{Remarque :}}
$f$ est convexe sur $\mathbb{R}^n$\\
$\Rightarrow \forall x \in \mathbb{R}^n$, $\partial f(x) \neq \emptyset$\\
et $\partial f(x)$ est un compact convexe de $\mathbb{R}^n$\\

\noindent Soit $k \in \mathbb{N}$
\begin{align*}
    ||x_{k+1} - x^*||_2^2 &= ||x_k - \alpha_k g_k - x^*||_2^2 \hspace{14.5cm}\\
    &= ||x_k - x^*||_2^2 - 2 \alpha_k g_k^T (x_k - x^*) + \alpha_k^2 ||g_k||_2^2\\
    &\textit{Texte manquant}\\
\end{align*}


On pose : $f^k_{best} = \underset{j \in \{1, \dots, k\}}{\text{min }} f(x_j)$\\
On a :\\
$2 \underset{l = 1}{\overset{k}{\sum}} \alpha_l (f(x_l) - f^*) \leq ||x_1 - x^*||_2^2 + \underset{l = 1}{\overset{k}{\sum}} ||gl||_2^2$\\
$\Rightarrow 0 \leq f^k_{best} - f(x^*) \leq \frac{||x_1 - x^*||_2^2 + \underset{l = 1}{\overset{k}{\sum}} ||gl||_2^2}{2 \underset{l = 1}{\overset{k}{\sum}} \alpha_l}$\\

Soit $R > 0$ tel que $||x_1 - x^*|| \leq R$\\

On a donc : $0 \leq f^k_{best} - f(x^*) \leq \frac{R^2 + \underset{l = 1}{\overset{k}{\sum}} ||gl||_2^2}{2 \underset{l = 1}{\overset{k}{\sum}} \alpha_l}$\\


On suppose de plus : $\exists a > 0$ tel que $\forall l \in \mathbb{N}^*$, $||gl|| \leq a$\\
$\Rightarrow 0 \leq f^k_{best} - f(x^*) \leq \frac{R^2 + a^2 \underset{l = 1}{\overset{k}{\sum}} \alpha_l^2}{2 \underset{l = 1}{\overset{k}{\sum}} \alpha_l}$\\



\noindent \underline{Quelle stratégie de pas $\alpha_l$ choisir ?}
\begin{itemize}
    \item Pas constant : $\forall l \in \{1, \dots, k\}$, $\alpha_l = \alpha > 0$\\
    
    On a donc : $0 \leq f^k_{best} - f(x^*) \leq \frac{R^2 + ka^2 \alpha^2}{2 k \alpha} \xrightarrow[k \rightarrow \infty]{} \frac{a^2 \alpha}{2}$\\
    $\Rightarrow f^k_{best} \in B_f(f(x^*), \frac{a^2 \alpha}{2})$\\

    \item $\forall l \in \{1, \dots, k\}, \alpha_l > 0$\\
    
    \item \begin{align*}
        \forall l \in \mathbb{N}, \alpha_l = \frac{\gamma_l}{gl} \qquad \text{ avec } \qquad &\gamma_l > 0 \hspace{14.5cm}\\
        &\underset{l \rightarrow \infty}{\lim} \gamma_l = 0\\
        &\underset{l = 1}{\overset{\infty}{\sum}} \gamma_l = \infty\\
    \end{align*}
\end{itemize}

\textit{Texte manquant}\\




\begin{align*}
    \forall k \geq N_1 + 1, \qquad 0 \leq f^k_{best} - f(x^*) &\leq \frac{R^2 + a^2 \underset{l = 1}{\overset{N_1}{\sum}} \alpha_l^2}{2 \underset{l = 1}{\overset{k}{\sum}} \alpha_l} + \frac{a^2 \underset{l = N_1 + 1}{\overset{k}{\sum}} \alpha_l^2}{2 \underset{l = 1}{\overset{k}{\sum}} \alpha_l}  \hspace{14.5cm}\\
    &= \frac{R^2 + a^2 \underset{l = 1}{\overset{N_1}{\sum}} \alpha_l^2}{2 \underset{l = 1}{\overset{k}{\sum}} \alpha_l} + \frac{\varepsilon}{2} \frac{\underset{l = N_1 + 1}{\overset{k}{\sum}} \alpha_l^2}{\underset{l = 1}{\overset{k}{\sum}} \alpha_l}
\end{align*}

De plus, $\underset{l = 1}{\overset{k}{\sum}} \alpha_l \xrightarrow[]{} \infty$\\

$\exists N_2 \in \mathbb{N}$ tel que $\forall k \geq N_2, \underset{l = 1}{\overset{k}{\sum}} \alpha_l^2 \geq \frac{R^2 + a^2 \underset{l = 1}{\overset{N_1}{\sum}} \alpha_l^2}{\varepsilon}$\\

D'où $\forall k \geq max(N_1 + 1, N_2)$, $0 \leq f^k_{best} - f(x^*) \leq \frac{\varepsilon}{2} + \frac{\varepsilon}{2} = \varepsilon$\\

Donc $f^k_{best} \xrightarrow[k \rightarrow \infty]{} f(x^*)$\\




\noindent\underline{\textbf{Remarque :}}
On s'intéresse à la stratégie de pas qui minimise $\psi(\alpha) = \frac{R^2 + a^2 ||\alpha ||_2^2}{2 ||\alpha||_1}$ avec $\alpha \in (\mathbb{R}_+^*)^k$\\
Avec une telle stratégie de pas, il pourra être nécessaire de faire $\mathcal{O}(\frac{1}{\varepsilon})$ itérations pour atteindre $|f^k_{best} - f(x^*)| \leq \varepsilon$\\
Pour $\varepsilon = 10^{-3}$, on a $10^6$ itérations : l'algorithme est très long dans le pire des cas.\\


\subsection{Algorithme du sous-gradient projeté}

Soit $f : \mathbb{R}^n \rightarrow \mathbb{R}$ convexe. On considère le problème suivant :\\
\begin{equation}
    \underset{x \in \mathcal{C}}{\text{min }} f(x) \quad \text{avec } \mathcal{C} \text{ partie convexe, fermée =, non-vide de } \mathbb{R}^n.
    \label{eq:pb}
\end{equation}

On suppose que (\ref{eq:pb}) admette au moins une solution $x^* \in \mathcal{C}$\\


\noindent \underline{\textbf{Remarque :}}
$\mathcal{C}$ partie convexe fermée non-vide de $\mathbb{R}^n$ :\\
La projection orthogonale sur $\mathcal{C}$, notée $\Pi_{\mathcal{C}}$, est $1$-lipschitzienne :
$$
\forall (x, y) \in (\mathbb{R}^n)^2, ||\Pi_{\mathcal{C}}(x) - \Pi_{\mathcal{C}}(y)||_2 \leq ||x - y||_2
$$

\noindent \underline{Idée}
Projection orthogonale sur $\mathcal{C}$ des itérations de l'algorithme du sous-gradient.\\


\noindent On a l'algorithme suivant :\\
\begin{algorithm}
    \SetAlgoLined
    \textbf{Entrées :} $x_0 \in \mathbb{R}^n$\\
    \textbf{Tant que}\\
    \begin{itemize}
        \item Calculer $g_k \in \partial f(x_k)$
        \item $x_{k+1} = \Pi_{\mathcal{C}}(x_k - \alpha_k g_k)$
    \end{itemize}
    \textbf{Fin Tant que}\\
\end{algorithm}



\noindent\underline{Y'a-t-il convergence ?}\\
Soit $z_{k+1} = x_k - \alpha_k g_k \in \mathbb{R}^n \quad$ avec $\quad g_k \in \partial f(x_k)$ et $\alpha_k > 0$

\begin{align*}
    ||z_{k+1} - x^*||_2^2 &= ||x_k - \alpha_k g_k - x^*||_2^2 \hspace{14.5cm}\\
    &\leq ||x_k - x^*||_2^2 - 2 \alpha_k (f(x_k) - f(x^*)) + \alpha_k^2 ||g_k||_2^2
\end{align*}
\begin{align*}
    \text{Or, } \quad ||x_k - x^*||_2^2 &= ||\Pi_{\mathcal{C}}(z_{k+1}) - \Pi_{\mathcal{C}}(x^*)||_2^2 \qquad \text{ car } x^* \in \mathcal{C} \Rightarrow \Pi_{\mathcal{C}}(x^*) = x^* \hspace{14.5cm}\\
    &\leq ||z_{k+1} - x^*||_2^2 \qquad \text{ car } \Pi_{\mathcal{C}} \text{ est } 1\text{-lipschitzienne}\\
    &\leq ||x_k - x^*||_2^2 - 2 \alpha_k (f(x_k) - f(x^*)) + \alpha_k^2 ||g_k||_2^2
\end{align*}


On retrouve les mêmes contraintes de convergence que pour l'algorithme du sous-gradient.\\
Notamment, on a besoin de $\mathcal{O}(\frac{1}{\varepsilon})$ itérations pour atteindre $|f^k_{best} - f(x^*)| \leq \varepsilon$\\


\noindent\underline{\textbf{Remarque :}}
$\Pi_{\mathcal{C}}$ peut être difficile à calculer en pratique selon ce qu'est $\mathcal{C}$\\



\subsection{Cas particulier : contraintes convexes d'inégalité}

On considère le problème suivant :\\
\begin{equation}
    \begin{cases}
        \underset{x \in \mathbb{R}^n}{\text{min }} f(x) \quad \text{avec } f \text{ et } (f_i)_{i \in \{1, \dots, p\}} \text{ convexe sur } \mathbb{R}^n\\
        \forall i \in \{1, \dots, p\}, f_i(x) \leq 0\\
    \end{cases}
    \label{eq:pb2}
\end{equation}

En posant $\mathcal{C} = \left\{ x \in \mathbb{R}^n \text{ tel que } \forall i \in \{1, \dots, p\}, f_i(x) \leq 0 \right\}$\\

Le problème (\ref{eq:pb2}) devient :\\
\begin{equation}
    \underset{x \in \mathcal{C}}{\text{min }} f(x) \qquad \text{ avec } \mathcal{C} \text{ convexe}
    \label{eq:pb3}
\end{equation}

Il est possible d'utiliser l'algorithme du sous-gradient projeté, mais $\Pi_{\mathcal{C}}$ peut être difficile à calculer.\\

\noindent \underline{Quelles autres stratégies ?}\\
On cherche un algorithme qui respecte les contraintes à chaque itérations.\\

\begin{algorithm}
    \SetAlgoLined
    \textbf{Entrées :} $x_0 \in \mathbb{R}^n$\\
    \textbf{Tant que}\\
    \begin{itemize}
        \item Calculer $g_k \in \mathbb{R}^n$ tel que : \begin{itemize}
            \item $g_k \in \partial f(x_k)$ si $x_k \in \mathcal{C}$
            \item $g_k \in \partial f_i(x_k)$ tel que $f_i(x_k) > 0$.\\
            \end{itemize}
        \item $x_{k+1} = x_k - \alpha_k g_k$
        \end{itemize}
    \textbf{Fin Tant que}\\
\end{algorithm}


On pose $f^k_{best} = \underset{j \in \{1, \dots, k\}}{\text{min }} f(x_j)$ avec $x_j \in \mathcal{C}$\\

On suppose de plus que : $\exists x_l \in \mathbb{R}^n$ tel que $\forall i \in \{1, \dots, p\}$, $f_i(x_l) < 0$ et $f(x_l) \neq f(x^*)$ (avec $x^*$ solution de (\ref{eq:pb3}))\\

Avec $(\alpha_l)$ tel que :
$
\begin{cases}
    \forall l \in \mathbb{N}^*, \alpha_l > 0\\
    \sum \alpha_l = \infty\\
    \sum \alpha_l^2 < \infty\\
\end{cases}
$\\

Alors $f^k_{best} \xrightarrow[k \rightarrow \infty]{} f(x^*)$\\

\noindent\underline{Preuve :}\\
Supposons le contraire : $\exists \varepsilon > 0$ tel que $\forall N \in \mathbb{N}$, $\exists k \geq N$ tel que $f^k_{best} \geq f(x^*) + \varepsilon$\\
En particulier, $f^N_{best} \geq f^k_{best} \geq f(x^*) + \varepsilon, \forall N \in \mathbb{N}$\\

Soit $\lambda \in [0, 1]$. On pose $\tilde{x} = (1 - \lambda) x^* + \lambda x_l \in \mathcal{C}$ (par convexité de $\mathcal{C}$)
\begin{align*}
    \text{Par convexité de } f, \qquad f(\tilde{x}) &\leq (1 - \lambda) f(x^*) + \lambda f(x_l) \hspace{14.5cm}\\
    &\leq f(x^*) + \lambda (f(x_l) - f(x^*))\\
\end{align*}

\textit{Texte manquant}\\


$\forall i \in \{1, \dots, p\}$, $f_i(\tilde{x}) \leq (1 - \lambda) f_i(x^*) + \lambda f_i(x_e) \quad \text{ par convexité de } f_i$\\
Or, $f_i(x^*) \leq 0$ car $x^* \in \mathcal{C}$ et $f_i(x_l) < 0$ par définition de $x_l$\\

\begin{align*}
    f_i(\tilde{x}) &\leq \lambda f_i(x_l) \hspace{14.5cm}\\
    &\leq \lambda \underset{i \in \{1, \dots, p\}}{\text{max }} f_i(x_l)\\
    &\leq - \mu \qquad \text{ avec } \mu = \lambda \underset{i \in \{1, \dots, p\}}{\text{max }} f_i(x_l) > 0\\
\end{align*}

D'où $\exists (\tilde{x}, \mu) \in \mathcal{C} \times \mathbb{R}_+^*$ tel que
$
\begin{cases}
    f(\tilde{x}) \leq f(x^*) + \frac{\varepsilon}{2}\\
    \forall i \in \{1, \dots, p\}, f_i(\tilde{x}) \leq - \mu\\
\end{cases}
$\\

\begin{align*}
    \forall k \in \mathbb{N}, \qquad ||x_{k+1} - \tilde{x}||_2^2 &= ||x_k - \alpha_k g_k - \tilde{x}||_2^2 \hspace{14.5cm}\\
    &= ||x_k - \tilde{x}||_2^2 - 2 \alpha_k g_k^T (x_k - \tilde{x}) + \alpha_k^2 ||g_k||_2^2\\
\end{align*}

\begin{itemize}
    \item 1er cas : $x_k \in \mathcal{C}$\\
    Alors $g_k \in \partial f(x_k)$ et $||x_{k+1} - \tilde{x}||_2^2 \leq ||x_k - \tilde{x}||_2^2 - 2 \alpha_k (f(x_k) - f(\tilde{x})) + \alpha_k^2 ||g_k||_2^2$\\
    Or, $x_k \in \mathcal{C}$ 
\end{itemize}

\textit{Texte manquant}\\




\section{Méthodes proximales}

\subsection{Fonction proximale}

Soit $h : \mathbb{R}^n \rightarrow \mathbb{R}$ convexe. On définit la fonction proximale de $h$, noté \textit{prox}$_h$, par :\\
\begin{equation}
    \forall x \in \mathbb{R}^n, \quad \text{prox}_h(x) = \underset{u \in \mathbb{R}^n}{\text{argmin }} \left\{ h(u) + \frac{1}{2} ||u - x||_2^2 \right\}
    \label{eq:prox}
\end{equation}

\noindent\underline{Exemple :}
\begin{enumerate}[label=\roman*)]
    \item $h = 0$ alors $\forall x \in \mathbb{R}^n$, $\text{prox}_h(x) = x$\\
    
    \item \textit{Texte manquant}\\

    \item $\forall x \in \mathbb{R}^n, h(x) = \lambda ||x||_1$ avec $\lambda > 0$\\
    Alors prox$_h$ est appelé \textit{Seuillage doux}, et est défini par :\\
    \begin{equation}
        \forall i \in \{1, \dots, n\}, \quad \left[ \text{prox}_h(x) \right]_i = \begin{cases}
            x_i - \lambda &\text{ si } x_i > \lambda\\
            0 &\text{ si } |x_i| \leq \lambda\\
            x_i + \lambda &\text{ si } x_i < -\lambda\\
        \end{cases}
    \end{equation}
\end{enumerate}


\subsection{Méthode du gradient proximal}
On s'intéresse au problème d'optimisation suivant :\\
\begin{equation}
    \underset{x \in \mathbb{R}^n}{\text{min }} f(x) = g(x) + h(x) 
    \label{eq:pb4}
\end{equation}
Avec : \begin{itemize}
    \item $g$ convexe et dérivable sur $\mathbb{R}^n$
    \item $h$ convexe, potentiellement non-lisse (mais telle que prox$_h$ soit facile à calculer)
\end{itemize}

\begin{algorithm}
    \SetAlgoLined
    \textbf{Entrées :} $x_0 \in \mathbb{R}^n$\\
    \textbf{Tant que}\\
    \begin{itemize}
        \item $x_{k+1} = \text{prox}_{\alpha_k} h (x_k - \alpha_k \nabla g(x_k))$       
    \end{itemize}
\end{algorithm}

et $\alpha_k$ obtenue depuis :
\begin{enumerate}[label=\roman*)]
    \item $\forall k \in \mathbb{N}, \alpha_k = \alpha > 0$ (pas constant)
    \item Recherche linéaire
\end{enumerate}

\noindent\underline{Exemple :}
\begin{enumerate}[label=\roman*)]
    \item $h = 0 : x_{k+1} = x_k - \alpha_k \nabla g(x_k)$ (méthode de descente de gradient)
    \item $h(x) = \begin{cases}
        1 \text{ si } x \in \mathcal{C} \qquad \text{ avec } \mathcal{C} \text{ convexe fermée non-vide de } \mathbb{R}^n\\
        + \infty \text{ sinon}\\
    \end{cases}$
    $x_{k+1} = \Pi_{\mathcal{C}}(x_k - \alpha_k \nabla g(x_k))$ (méthode du gradient projeté)
    \item $h(x) = \lambda ||x||_1$ avec $\lambda > 0$\\
    Seuillage doux pour minimiser $f(x) = g(x) + \lambda ||x||_1$\\
\end{enumerate}


\noindent\underline{\textbf{Remarque :}}
\begin{enumerate}[label=\roman*)]
    \item \begin{align*}
        x_{k+1} &= \underset{u \in \mathbb{R}^n}{\text{argmin }} \left\{\alpha_k h(u) + \frac{1}{2} ||u - x_k + \alpha_k \nabla g(x_k)||_2^2 \right\}\\
        &= \underset{u \in \mathbb{R}^n}{\text{argmin }} \left\{h(u) + \frac{1}{2 \alpha_k} ||u - x_k||_2^2 + 2 \alpha_k \nabla g(x_k)^T (u - x_k) + \alpha^2 ||\nabla g(x_k)||_2^2 \right\}\\
        &= \underset{u \in \mathbb{R}^n}{\text{argmin }} \left\{h(u) + g(x_k) + \nabla g(x_k)^T (u - x_k) + \frac{1}{2 \alpha_k} ||u - x_k||_2^2 + \frac{\alpha_k}{2} ||\nabla g(x_k)||_2^2 \right\}\\
        &= \underset{u \in \mathbb{R}^n}{\text{argmin }} \left\{h(u) + g(x_k) + \nabla g(x_k)^T (u - x_k) + \frac{1}{2 \alpha_k} ||u - x_k ||_2^2 \right\}\\
    \end{align*}
    Avec : $g(x_k) + \nabla g(x_k)^T (u - x_k) + \frac{1}{2 \alpha_k} ||u - x_k||_2^2$ LE modèle quadratique \textit{dégénéré} de $g$ en $x_k$\\
    \item $u = \text{prox } h(x) \Leftrightarrow 0 \in \partial \rho (u)$ avec $\rho (u) = h(u) + \frac{1}{2} ||u - x||_2^2$\\
    $\Leftrightarrow 0 \in \partial h(u) + \left\{u - x \right\}$\\
    $\Leftrightarrow x - u \in \partial h(u)$\\
    \item La méthode du gradient proximal peut se réécrire :\\
    $x_{k+1} = x_k + \alpha_k G_{\alpha_k} (x_k)$ avec $G_{\alpha_k}(x) = \frac{1}{\alpha} (x - \text{prox } \alpha h(x - \alpha \nabla g(x)))$\\
    
    \begin{align*}
        \text{En effet, } \qquad &x_k - \alpha_k G_{\alpha_k}(x_k) = \text{prox } \alpha_k h(x_k - \alpha_k \nabla g(x_k))\\
        &\Leftrightarrow x_k \alpha_k \nabla g(x_k) - x_k + \alpha_k G_{\alpha_k}(x_k) \in \partial \alpha_k h(x_k - \alpha_k G_{\alpha_k}(x_k)) \qquad \text{ par ii)} \hspace{14.5cm}\\
        &\Leftrightarrow \alpha_k \left[ G_{\alpha_k}(x_k) - \nabla g(x_k) \right] \in \partial \alpha_k h(x_k - \alpha_k G_{\alpha_k}(x_k))\\
        &\Leftrightarrow G_{\alpha_k}(x_k) - \nabla g(x_k) \in \partial h(x_k - \alpha_k G_{\alpha_k}(x_k))\\
        &\Leftrightarrow G_{\alpha_k}(x_k) \in \partial h(x_k - \alpha_k G_{\alpha_k}(x_k)) + \left\{ \nabla g(x_k) \right\}
    \end{align*}
    De plus, $G_{\alpha_k}(x_k) = 0 \Leftrightarrow 0 \in \partial h(x_k) + \left\{ \nabla g(x_k) \right\} \Leftrightarrow x_k$ est un minimum de $f$\\

    D'où : tout point fixe de la suite des itérés du gradient proximal est un minimum de $f$\\    
\end{enumerate}


\subsection{Recherche linéaire (choix des pas) et convergence de l'algorithme}


\begin{algorithm}
    Recherche linéaire : backtracking et condition d'arrêt modifiée\\
    \SetAlgoLined
    \textbf{Entrées :} $\alpha_0 > 0$, $\beta \in ]0, 1[$, $x_k \in \mathbb{R}^n$, $\nabla g(x_k) \in \mathbb{R}^n$\\
    \textbf{Tant que}\\
    \begin{itemize}
        \item $\alpha_{l+1} = \beta \alpha_l$
        \item Critère d'arrêt : $g(x_k - \alpha_l G_{\alpha_l}(x_k)) \leq g(x_k) - \alpha_l G_{\alpha_l}(x_k)^T \nabla g(x_k) + \frac{\alpha_l}{2} ||G_{\alpha_l}(x_k)||_2^2$\\
    \end{itemize}
    \textbf{Fin Tant que}\\
\end{algorithm}


\propriete{}{
    On suppose que $\nabla g$ est $L$-lipschitzienne avec $L > 0$ :\\
    $\forall (x, y) \in (\mathbb{R}^n)^2, ||\nabla g(x) - \nabla g(y)||_2 \leq L ||x - y||_2$\\
    Alors l'algorithme de recherche linéaire s'arrête sur $\alpha_l \geq \text{min } (\alpha_0, \frac{\beta}{L}) := \alpha_{min}$
}

\noindent\underline{Preuve :}\\
$\nabla g$ est $L$-lipschitzienne $\Rightarrow \forall (x, y) \in (\mathbb{R}^n)^2, g(y) \leq g(x) + \nabla g(x)^T (y - x) + \frac{L}{2} ||y - x||_2^2$\\
$\Rightarrow g(x_k - \alpha_l G_{\alpha_l}(x_k)) \leq g(x_k) - \alpha_l G_{\alpha_l}(x_k)^T \nabla g(x_k) + \frac{L}{2} \alpha_l^2 ||G_{\alpha_l}(x_k)||_2^2$\\

\noindent La condition est valide pour tout $\alpha_l \in ]0, \frac{1}{L}[$\\
Donc à l'arrêt de la recherche linéaire, $\alpha_l \geq \text{min } (\alpha_0, \frac{\beta}{L})$\\


\propriete{}{
    On suppose que $\nabla g$ est $L$-lipschitzienne. A l'arrêt de la recherche linéaire ($\alpha_k, G_{\alpha_k}(x_k)$), on a :\\
    $\forall z \in \mathbb{R}^n, \quad f(x_k - \alpha_k G_{\alpha_k}(x_k)) \leq f(z) + G_{\alpha_k}(x_k)^T (x_k - z) - \frac{\alpha_k}{2} ||G_{\alpha_k}(x_k)||_2^2$
}

\noindent\underline{Preuve :}
Par définition du critère d'arrêt de la recherche linéaire :
\begin{align*}
    g(x_k - \alpha_k G_{\alpha_k}(x_k)) &\leq g(x_k) - \alpha_k G_{\alpha_k}(x_k)^T \nabla g(x_k) + \frac{\alpha_k}{2} ||G_{\alpha_k}(x_k)||_2^2 \hspace{14.5cm}\\
    &\leq g(x_k) + \alpha_k G_{\alpha_k}(x_k)^T (G_{\alpha_k}(x_k) - \nabla g(x_k)) - \frac{\alpha_k}{2} ||G_{\alpha_k}(x_k)||_2^2
\end{align*}

On a : $G_{\alpha_k}(x_k) - \nabla g(x_k) \in \partial h(x_k - \alpha_k G_{\alpha_k}(x_k))$\\
Donc : $\forall z \in \mathbb{R}^n, h(z) \geq [G_{\alpha_k}(x_k) - \nabla g(x_k)]^T (z - x_k + \alpha_k G_{\alpha_k}(x_k)) + h(x_k - \alpha_k G_{\alpha_k}(x_k))$\\

De plus,
\begin{align*}
    g(x_k - \alpha_k G_{\alpha_k}(x_k)) + h(x_k - \alpha_k G_{\alpha_k}(x_k)) \leq g(x_k) &+ h(x_k - \alpha_k G_{\alpha_k}(x_k))\\
    &+ \alpha_k G_{\alpha_k}(x_k)^T (G_{\alpha_k}(x_k) - \nabla g(x_k))\\
    &- \frac{\alpha_k}{2} ||G_{\alpha_k}(x_k)||_2^2 \hspace{14.5cm}\\
    \Leftrightarrow f(x_k - \alpha_k G_{\alpha_k}(x_k)) \leq g(x_k) &+ h(x_k - \alpha_k G_{\alpha_k}(x_k))\\
    &+ \alpha_k G_{\alpha_k}(x_k)^T (G_{\alpha_k}(x_k) - \nabla g(x_k))\\
    &- \frac{\alpha_k}{2} ||G_{\alpha_k}(x_k)||_2^2
\end{align*}

Or, $h(x_k - \alpha_k G_{\alpha_k}(x_k)) \leq h(z) + [G_{\alpha_k}(x_k) - \nabla g(x_k)]^T (x_k - z - \alpha_k G_{\alpha_k}(x_k))$\\
\begin{align*}
    \Rightarrow f(x_k - \alpha_k G_{\alpha_k}(x_k)) \leq g(x_k) &+ h(z) + \alpha_k G_{\alpha_k}(x_k)^T (G_{\alpha_k}(x_k) - \nabla g(x_k))\\
    &+ [G_{\alpha_k}(x_k) - \nabla g(x_k)]^T (x_k - z - \alpha_k G_{\alpha_k}(x_k))\\
    &- \frac{\alpha_k}{2} ||G_{\alpha_k}(x_k)||_2^2 \hspace{14.5cm}\\
\end{align*}

$f(x_k - \alpha_k G_{\alpha_k}(x_k)) \leq g(x_k) + h(z) + [G_{\alpha_k}(x_k) - \nabla g(x_k)]^T (x_k - z) - \frac{\alpha_k}{2} ||G_{\alpha_k}(x_k)||_2^2$\\

Or, $g$ est convexe et dérivable sur $\mathbb{R}^n$ d'où :
$\forall z \in \mathbb{R}^n, g(z) \geq \nabla g(x_k)^T (z - x_k) + g(x_k)$\\
$\Rightarrow g(x_k) \leq g(z) + \nabla g(x_k)^T (x_k - z)$\\


\textit{Texte manquant}\\


\end{document}