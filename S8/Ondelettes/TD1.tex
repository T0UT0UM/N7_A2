\documentclass[12pt,a4paper]{article}
\usepackage[utf8]{inputenc} 
\usepackage[T1]{fontenc}		       
\usepackage{lmodern}			       
\usepackage{babel} 
\usepackage{amsmath}
\usepackage{amsfonts}
\usepackage{amssymb}
\usepackage{graphicx}
\usepackage{xcolor}
\usepackage{mathtools}
\usepackage{fancyhdr}
\usepackage{enumitem}
\usepackage{tcolorbox}
\usepackage{stmaryrd}
\usepackage{dsfont}
\usepackage{tikz}
\usepackage[linesnumbered,ruled,vlined]{algorithm2e}
\usepackage[text={15cm,24.5cm},centering]{geometry}

% Définir le texte affiché en fin de page
\pagestyle{fancy}
\fancyhf{}  % Clear the default headers and footers
\rfoot{\hrule
    \vspace{0.3cm}
    \noindent\textsf{Félix de Brandois}
    \hfill \thepage
}
\renewcommand{\headrulewidth}{0pt}

% Style de l'entete
\newcommand{\entete}{
    \noindent\textbf{INSA - ModIA, 4$^e$ année.}
    \hfill \textbf{Années 2023-2024}
    
    \begin{center}
        \textbf{\LARGE TD : Transformée de Haar}
    \end{center}
}





\begin{document}

\entete

\vspace{0.5cm}


On note $\Phi$ la fonction définie sur $\mathbb{R}$ indicatrice de l'intervalle $[0,1[$ :
$$
\Phi(x) = \begin{cases}
1 & \text{si } x \in [0,1[ \\
0 & \text{sinon}
\end{cases}
$$

Pour tout $j \leq 0$ et $k \in \llbracket 0, 2^{-j} - 1 \rrbracket$, on définit la fonction $\phi_{j,k}$ sur $\mathbb{R}$ par :
$$
\phi_{j,k}(x) = 2^{-\frac{j}{2}} \Phi(2^{-j} x - k)
$$

On note $V_j = \text{Vect}(\phi_{j,k})_{0 \leq k \leq 2^{-j} - 1}$ l'espace vectoriel engendré par les fonctions $\phi_{j,k}$.
On notera $\Omega$ l'ensemble des couples $(j,k)$ tels que $j \leq 0$ et $k \in \llbracket 0, 2^{-j} - 1 \rrbracket$.


\section{Propriété des fonctions $\phi_{j,k}$ pour les bases de Haar}

\begin{enumerate}
    \color{black}
    \item Montrer que pour tout couple $(j,k) \in \Omega$
    $$
    \Phi_{j,k}(x) = \begin{cases}
    2^{-\frac{j}{2}} & \text{si } x \in [k2^{j}, (k+1)2^{j}[ \\
    0 & \text{sinon}
    \end{cases}
    $$

    \color{blue}
    Soit $(j,k) \in \Omega$ et $x \in \mathbb{R}$.
    \begin{align*}
        & \qquad \Phi_{j,k}(x) = 2^{-\frac{j}{2}} \hspace{14.5cm} \\
        & \Leftrightarrow \Phi(2^j x - k) = 1 \\
        & \Leftrightarrow 0 \leq 2^j x - k < 1 \\
        & \Leftrightarrow k2^j \leq x < (k+1)2^j \\
    \end{align*}
     

    \color{black}
    \item Expliciter et représenter graphiquement les fonctions suivantes :
    $$
    \Phi_{0,0}, \quad \Phi_{-1,0}, \quad \Phi_{-1,1}, \quad \Phi_{-2,0}, \quad \Phi_{-2,3}
    $$

    \color{blue}
    \begin{tikzpicture}
        \draw[->] (-3,0) -- (3,0) node[right] {$x$};
        \draw[->] (0,-0.5) -- (0,2) node[above] {$y$};
        \draw[domain=-3:3,smooth,variable=\x,black] plot ({\x},{\x > 0 ? 1 : 0});
        \draw[domain=-3:3,smooth,variable=\x,blue] plot ({\x},{\x > 0 ? 1 : 0});
        \draw[domain=-3:3,smooth,variable=\x,red] plot ({\x},{\x > 1 ? 1 : 0});
        \draw[domain=-3:3,smooth,variable=\x,green] plot ({\x},{\x > 0 ? 1 : 0});
        \draw[domain=-3:3,smooth,variable=\x,orange] plot ({\x},{\x > 3 ? 1 : 0});
    \end{tikzpicture}


    \color{black}
    \item Montrer que $(\Phi_{j,k})_{0 \leq k \leq 2^{-j} - 1}$ est une base orthonormée de $V_j$ pour le produit scalaire défini par :
    \begin{equation}
        \langle f, g \rangle = \int_{0}^{1} f(t) g(t) dt
    \end{equation}

    \color{blue}
    Soient $k, k' \in \llbracket 0, 2^{-j} - 1 \rrbracket$.
    \begin{align*}
        & \qquad \langle \Phi_{j,k}, \Phi_{j,k'} \rangle = 0 \hspace{14.5cm} \\
        & \Leftrightarrow \int_{0}^{1} \Phi_{j,k}(t) \Phi_{j,k'}(t) dt = 0 \\
        & \Leftrightarrow [k2^j, (k+1)2^j[ \cap [k'2^j, (k'+1)2^j[ = \emptyset \\
        & \Leftrightarrow (k+1)2^j \leq k'2^j \text{ ou } (k'+1)2^j \leq k2^j \\
        & \Leftrightarrow k + 1 \leq k' \text{ ou } k' + 1 \leq k \\
        & \Leftrightarrow k \neq k'
    \end{align*}

    De plus, pour tout $k \in \llbracket 0, 2^{-j} - 1 \rrbracket$,
    \begin{align*}
        ||\Phi_{j,k}||^2 &= \int_{0}^{1} \Phi_{j,k}(t) dt \hspace{14.5cm} \\
        &= \int_{0}^{1} 2^{-\frac{j}{2}} \Phi(2^{-j} t - k) dt \\
        &= 2^{-j} \int_{k2^j}^{(k+1)2^j} 1 dt \\
        &= 2^{-j} \times 2^j \\
        &= 1
    \end{align*}
    Ainsi, $(\Phi_{j,k})_{0 \leq k \leq 2^{-j} - 1}$ est une base orthonormée de $V_j$.\\


    \color{black}
    \item Quelle est la dimension de $V_j$ ?
    
    \color{blue}
    Pour $j$ fixé, $k \in \llbracket 0, 2^{-j} - 1 \rrbracket$. Donc la dimension de $V_j$ est $2^{-j}$.\\
    

    \color{black}
    \item Montrer que pour tout couple $(j,k) \in \Omega$, on a :
    \begin{equation}
        \Phi_{j,k} = \frac{\sqrt{2}}{2} \Phi_{j-1, 2k} + \frac{\sqrt{2}}{2} \Phi_{j-1, 2k+1}
        \label{eq:decomposition}
    \end{equation}

    \color{blue}
    Soit $(j,k) \in \Omega$.
    \begin{align*}
        \frac{\sqrt{2}}{2} (\Phi_{j-1, 2k} + \Phi_{j-1, 2k+1})(x) &= \begin{cases}
            \frac{\sqrt{2}}{2} 2^{-\frac{j-1}{2}} & \text{si } x \in [2k2^{j-1}, (2k+1)2^{j-1}[ \\
            \frac{\sqrt{2}}{2} 2^{-\frac{j-1}{2}} & \text{si } x \in [(2k+1)2^{j-1}, (2k+2)2^{j-1}[ \\
            0 & \text{sinon}
        \end{cases}\\
        &= \begin{cases}
            2^{-\frac{j}{2}} & \text{si } x \in [k2^{j}, (k+1)2^{j}[ \\
            0 & \text{sinon}
        \end{cases} \\
        &= \Phi_{j,k}(x)\\
    \end{align*}


    \color{black}
    \item En déduire que $\forall j \leq 0$, $V_j \subset V_{j-1}$.
    
    \color{blue}
    D'après la question précédente, pour tout $(j,k) \in \Omega$, $\Phi_{j,k} \in V_{j-1}$.\\
    Donc $V_j \subset V_{j-1}$.\\


    \color{black}
    \item En déduire de (\ref{eq:decomposition}) que pour tout $t \in \mathbb{R}$, on a :
    \begin{equation}
        \frac{\sqrt{2}}{2} \Phi(t) = \frac{\sqrt{2}}{2} \Phi(2t) + \frac{\sqrt{2}}{2} \Phi(2t - 1)
        \label{eq:decomposition_phi}
    \end{equation}

    \color{blue}
    Soient $j = k = 0$.
    \begin{align*}
        \qquad &\Phi_{0,0} = \frac{\sqrt{2}}{2} \Phi_{-1,0} + \frac{\sqrt{2}}{2} \Phi_{-1,1} \hspace{14.5cm} \\
        &\Leftrightarrow \Phi(t) = \Phi(2t) + \Phi(2t - 1) \\
    \end{align*}


    \color{black}
    On définit la suite $(h_n)_{n \in \mathbb{Z}}$ par :
    $$
    h_n = \begin{cases}
        \frac{\sqrt{2}}{2} & \text{si } n = 0 \text{ ou } n = 1 \\
        0 & \text{sinon}
    \end{cases}
    $$

    \item Montrer qu'on peut réécrire (\ref{eq:decomposition_phi}) sous la forme :
    \begin{equation}
        \frac{\sqrt{2}}{2} \Phi(t) = \sum_{n \in \mathbb{Z}} h_n \Phi(2t - n)
        \label{eq:decomposition_phi_somme}
    \end{equation}

    \color{blue}
    \begin{align*}
        \sum_{n \in \mathbb{Z}} h_n \Phi(2t - n) &= h_0 \Phi(2t) + h_1 \Phi(2t - 1) \hspace{14.5cm} \\
        &= \frac{\sqrt{2}}{2} \Phi(t)
    \end{align*}


    \color{black}
    \item En calculant la transformée de Fourier des deux membres de (\ref{eq:decomposition_phi_somme}), montrer que pour tout $\omega \in \mathbb{R}$, on a :
    \begin{equation}
        \hat{\Phi}(2\omega) = \frac{\sqrt{2}}{2} \hat{h}(\omega) \hat{\Phi}(\omega)
        \label{eq:decomposition_phi_fourier}
    \end{equation}


\end{enumerate}


\end{document}