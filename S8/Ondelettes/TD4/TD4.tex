\documentclass[12pt,a4paper]{article}
\usepackage[utf8]{inputenc} 
\usepackage[T1]{fontenc}		       
\usepackage{lmodern}			       
\usepackage{babel} 
\usepackage{amsmath}
\usepackage{amsfonts}
\usepackage{amssymb}
\usepackage{graphicx}
\usepackage{xcolor}
\usepackage{mathtools}
\usepackage{fancyhdr}
\usepackage{enumitem}
\usepackage{tcolorbox}
\usepackage{stmaryrd}
\usepackage{dsfont}
\usepackage{tikz}
\usepackage[upgreek]{txgreeks}
\usepackage[linesnumbered,ruled,vlined]{algorithm2e}
\usepackage[text={15cm,24.5cm},centering]{geometry}

% Définir le texte affiché en fin de page
\pagestyle{fancy}
\fancyhf{}  % Clear the default headers and footers
\rfoot{\hrule
    \vspace{0.3cm}
    \noindent\textsf{Félix de Brandois}
    \hfill \thepage
}
\renewcommand{\headrulewidth}{0pt}

% Style de l'entete
\newcommand{\entete}{
    \noindent\textbf{INSA - ModIA, 4$^e$ année.}
    \hfill \textbf{Années 2023-2024}
    
    \begin{center}
        \textbf{\LARGE TD : Ondes et moment nuls}
    \end{center}
}





\begin{document}

\entete

\vspace{0.5cm}

\begin{enumerate}
    \item Montrer que $\int_{\mathbb{R}} \Psi(t) \, dt = 0 \Leftrightarrow \hat{\Psi}(0) = 0$. \\
    Puis que $\Psi$ admet $2$ moments nulls si et seulement si $\hat{\Psi}(0) = \hat{\Psi}(1) = 0$. \\


    \color{blue}

    Par définition de la transformée de Fourier, on a : $\hat{\Psi}(w) = \int_{\mathbb{R}} \Psi(t) e^{-i wt} \, dt$. \\
    Donc $\hat{\Psi}(0) = \int_{\mathbb{R}} \Psi(t) \, dt$. \\

    On a : $-it \Psi(t) e^{-iwt} \in L^1(\mathbb{R})$ donc $\hat{\Psi}'(w) = -i \int_{\mathbb{R}} t \Psi(t) e^{-iwt} \, dt$. \\
    Donc $\hat{\Psi}'(0) = 0 \Leftrightarrow \int_{\mathbb{R}} t \Psi(t) \, dt = 0$. \\


    \color{black}

    \item Rappeler $\hat{h}(0)$ et $\hat{\phi}(0)$. Déduire que toutes les ondelettes ont au moins un moment nul. \\
    

    \color{blue}

    $h$ vérifie Mallat-Meyer, donc
    \begin{itemize}
        \item $\hat{h}(0) = 0$
        \item $\hat{\phi}(w) = \prod_{k=1}^{+\infty} \frac{\hat{h}(2^{-k}w)}{\sqrt{2}} \Rightarrow \hat{\phi}(0) = 1$
    \end{itemize}

    On a également : $\hat{\Psi}(2w) = \frac{1}{\sqrt{2}} \hat{g}(w) \hat{\phi}(w)$. avec $\hat{g}(w) = e^{-i w}\hat{h}(w+\pi)$. \\

\end{enumerate}




\end{document}